\documentclass{beamer}

\usepackage[utf8]{inputenc}
\usepackage[spanish]{babel}
\usepackage[outputdir=.build]{minted}
\usepackage{hyperref}
\usepackage{graphicx}

\hypersetup{
    colorlinks = true
}

\usetheme{Madrid}

%Information to be included in the title page:
\title{Taller de Física Computacional (Serie S)}
\subtitle{Diseño y construcción}
\author{Cristián G. Sánchez y Carlos J. Ruestes}
\date{2021}

\begin{document}

\frame{\titlepage}

%%%%%%%%%%%%%%%%%%%%%%%%%%%%%%%%%%%%%%%%%%%%%%%%%%%%%%%%%%
\begin{frame}[fragile]
    \frametitle{Diseño o construcción de programas}
    \begin{itemize}
        \item Analizar el problema: ¿Cuál es el problema a resolver? Incluyendo el contexto en el que se usará.
        \item Especificar la solución: ¿Qué debe hacer el programa? Sin importar el cómo.
        \item Diseñar la solución: ¿Cómo vamos a resolver el problema? Dividimos en sub-problemas, etc..
        \item Implementar el diseño: Traducir el diseño a un lenguaje de progrmación.
        \item Probar el programa: Diseñar pruebas para cada una de las partes por separado asi como su correcta integración.
        Depurar el código hasta que todas las pruebas pasen.
        \item Mantener el programa: Realizar los cambios necesarios en respuesta e nuevas demandas.
    \end{itemize}
\end{frame}



%%%%%%%%%%%%%%%%%%%%%%%%%%%%%%%%%%%%%%%%%%%%%%%%%%%%%%%%%%%
\begin{frame}
\frametitle{Síntesis y recursos:}

\begin{itemize}
\item \href{https://numpy.org/doc/stable/}{Documentación de NumPy}
\item \href{https://numpy.org/doc/stable/reference/routines.math.html}{Funciones matemáticas en NumPy}

\end{itemize}
\end{frame}

\end{document}

\documentclass{beamer}

\usepackage[utf8]{inputenc}
\usepackage[spanish]{babel}
\usepackage[outputdir=.build]{minted}
\usepackage{hyperref}
\usepackage{graphicx}

\hypersetup{
    colorlinks = true
}

\usetheme{Madrid}

%Information to be included in the title page:
\title{Taller de Física Computacional}
\subtitle{Literales y Operadores aritméticos}
\author{Cristián G. Sánchez y Carlos J. Ruestes}
\date{2021}

\begin{document}

\frame{\titlepage}

%%%%%%%%%%%%%%%%%%%%%%%%%%%%%%%%%%%%%%%%%%%%%%%%%%%%%%%%%%
\begin{frame}
\frametitle{Aclaración terminológica}
\begin{block}{{\em Aclaración terminológica}}
La documentación oficial de Python está en inglés. Vamos a tratar de traducir las palabras importantes dentro de lo posible. A veces utilizaremos directamente en inglés para simplificarnos la existencia.
\end{block}
\end{frame}

%%%%%%%%%%%%%%%%%%%%%%%%%%%%%%%%%%%%%%%%%%%%%%%%%%%%%%%%%%
\begin{frame}
\frametitle{Tokens}
\begin{block}{{\em Tokens}}
(Simplificando) un {\bf programa} escrito en Python es leído por el intérprete como una secuencia de {\em tokens}.
\end{block}
\begin{block}{Lista de {\em tokens}:}
NUEVA LÍNEA, INDENTACIÓN, DEDENTACIÓN, identificadores, palabras clave, literales, operadores y delimitadores
\end{block}
\end{frame}

%%%%%%%%%%%%%%%%%%%%%%%%%%%%%%%%%%%%%%%%%%%%%%%%%%%%%%%%%%%
\begin{frame}[fragile]
\frametitle{Palabras clave}
\begin{block}{{Palabras clave}}
%\begin{onlyenv}<2>
\begin{minted}{python}
False      await      else       import     pass
None       break      except     in         raise
True       class      finally    is         return
and        continue   for        lambda     try
as         def        from       nonlocal   while
assert     del        global     not        with
async      elif       if         or         yield
\end{minted}
%\end{onlyenv}
\end{block}
\end{frame}

%%%%%%%%%%%%%%%%%%%%%%%%%%%%%%%%%%%%%%%%%%%%%%%%%%%%%%%%%%%
\begin{frame}[fragile]
\frametitle{Operadores}
\begin{block}{{Operadores}}
%\begin{onlyenv}<2>
\begin{minted}{python}
+       -       *       **      /       //      %      @
<<      >>      &       |       ^       ~       :=
<       >       <=      >=      ==      !=
\end{minted}
%\end{onlyenv}
\end{block}
\end{frame}

%%%%%%%%%%%%%%%%%%%%%%%%%%%%%%%%%%%%%%%%%%%%%%%%%%%%%%%%%%%
\begin{frame}[fragile]
\frametitle{Delimitadores}
\begin{block}{{Delimitadores}}
%\begin{onlyenv}<2>
\begin{minted}{python}
(       )       [       ]       {       }
,       :       .       ;       @       =       ->
+=      -=      *=      /=      //=     %=      @=
&=      |=      ^=      >>=     <<=     **=
\end{minted}
%\end{onlyenv}
\end{block}
\end{frame}

%%%%%%%%%%%%%%%%%%%%%%%%%%%%%%%%%%%%%%%%%%%%%%%%%%%%%%%%%%
\begin{frame}[fragile]
\frametitle{Identificadores o nombres}
\begin{block}{{\em Identificadores o nombres}}
\alert{para más adelante}
\end{block}
\end{frame}

%%%%%%%%%%%%%%%%%%%%%%%%%%%%%%%%%%%%%%%%%%%%%%%%%%%%%%%%%%
\begin{frame}[fragile]
\frametitle{Literales}
\begin{block}{Literales}
Los {\em literales} son una notación para valores {\bf constantes} de {\bf algunos} tipos incorporados en el lenguaje.
\end{block}
\end{frame}

%%%%%%%%%%%%%%%%%%%%%%%%%%%%%%%%%%%%%%%%%%%%%%%%%%%%%%%%%%
\begin{frame}
\frametitle{Tokens}
\begin{block}{{\em Tokens}}
(Simplificando) un {\bf programa} escrito en Python es leído por el intérprete como una secuencia de {\em tokens}.
\end{block}
\begin{block}{Lista de {\em tokens}:}
NUEVA LÍNEA, INDENTACIÓN, DEDENTACIÓN, identificadores, palabras clave, literales, operadores y delimitadores
\end{block}
\end{frame}

%%%%%%%%%%%%%%%%%%%%%%%%%%%%%%%%%%%%%%%%%%%%%%%%%%%%%%%%%%
\begin{frame}[fragile]
\frametitle{Literales Numéricos}
\begin{block}{Enteros}
\begin{itemize}
\item Decimales: \mintinline{python}{123431235} o \mintinline{python}{123_431_235} 
\item Binarios: \mintinline{python}{0b0101010101}  o \mintinline{python}{0B0_1010_101_01} (puede ser 
\item Octales: \mintinline{python}{0o01234567} o \mintinline{python}{0O01_234_567}
\item Hexadecimales: \mintinline{python}{0xFFFFFFFF} o \mintinline{python}{0XFF_FF_FF_FF}  
\end{itemize}
Notas: Los caracteres {\tt b,o} y {\tt f} pueden ser mayúscula o minúscula. El guión bajo puede usarse para agrupar dígitos de forma arbitraria. Los enteros en python (versión 3 en adelante) no son acotados.
\end{block}
\end{frame}

%%%%%%%%%%%%%%%%%%%%%%%%%%%%%%%%%%%%%%%%%%%%%%%%%%%%%%%%%%
\begin{frame}[fragile]
\frametitle{Literales Numéricos}
\begin{block}{Punto Flotante}
\begin{itemize}
\item \mintinline{python}{3.14} o \mintinline{python}{10.} o  \mintinline{python}{10.0}  o \mintinline{python}{31.4e-1} o  \mintinline{python}{31.4e5} 

\end{itemize}
Nota: Los las cotas para los números de punto flotante dependen de la implementación.
\end{block}
\begin{block}{Imaginarios}
\begin{itemize}
\item \mintinline{python}{3.14j} o \mintinline{python}{10.j} o  \mintinline{python}{10.0j}  o \mintinline{python}{31.4e-1j} o  \mintinline{python}{31.4e5j} 
\end{itemize}
Nota: Los números complejos se construyen sumando un imaginario a un punto flotante real.
\end{block}
\end{frame}

%%%%%%%%%%%%%%%%%%%%%%%%%%%%%%%%%%%%%%%%%%%%%%%%%%%%%%%%%%
\begin{frame}[fragile]
\frametitle{Literales de caracteres}
\begin{block}{Cadenas de caracteres}
\begin{itemize}
\item \mintinline{python}{"Esto es una cadena"} o \mintinline{python}{'Esto es una cadena'}
\item Se pueden construir cadenas multilínea con triples comillas:
\begin{minted}{python}
"""Esto es
una cadena"""
\end{minted}
o 
\begin{minted}{python}
'''Esto también es
una cadena'''
\end{minted}
\item Las {\em secuencias de escape} \mintinline{python}{\n \t \\ \' \"} se utilizan para insertar una nueva línea, un tabulador, una barra y los dos tipos de comillas respectivamente, dentro de una cadena de caracteres.
\end{itemize}
\end{block}

\end{frame}

%%%%%%%%%%%%%%%%%%%%%%%%%%%%%%%%%%%%%%%%%%%%%%%%%%%%%%%%%%
\begin{frame}[fragile]
\frametitle{Literales de caracteres}
\begin{block}{Cadenas de formato}
\begin{itemize}
\item Las cadenas de formato se utilizan para construir cadenas que contienen valores de \alert{expresiones} que son evaluadas y a las que se les da formato en el momento de construcción de la cadena. 
\item Las cadenas de formato están precedidas por una {\tt f} seguida de alguna comilla y contienen las expresiones a ser evaluadas y una especificación de formato entre llaves. Siguen algunos ejemplos:
\item  \mintinline{python}{f" uno y un tercio con tres decimales es {1 + 1/3:.3f}"}
\item  \mintinline{python}{f" mil en un espacio de 10 caracteres es {10*10*10:10d}"}
\item La expresión de formato tiene muchas posibilidades, la documentación puede encontrarse en el \href{https://docs.python.org/3/library/string.html}{manual}.
\end{itemize}
\end{block}

\end{frame}

%%%%%%%%%%%%%%%%%%%%%%%%%%%%%%%%%%%%%%%%%%%%%%%%%%%%%%%%%%
\begin{frame}[fragile]
\frametitle{Operadores aritméticos}
Para poder empezar a hacer ``algo'' usando Python introducimos ahora los operadores aritméticos:
\begin{block}{Operadores aritméticos}
\begin{itemize}
\item \mintinline{python}{+ }: Suma como binario o elemento positivo como unario.
\item \mintinline{python}{- }: Resta como binario o negativo como unario.
\item \mintinline{python}{* }: Multiplicación, es binario.
\item \mintinline{python}{/ }: División, es binario.
\item \mintinline{python}{// }: División entera, es binario.
\item \mintinline{python}{% }: Módulo (resto de la división entera), es binario.
\item \mintinline{python}{** }: Exponenciación, es binario
\item Se pueden utilizar paréntesis dentro de una expresión para asegurar el orden de evaluación.
\end{itemize}
Estos operadores son un subconjunto de los operadores que listamos anteriormente al hablar de {\em tokens}.
\end{block}
\end{frame}

%%%%%%%%%%%%%%%%%%%%%%%%%%%%%%%%%%%%%%%%%%%%%%%%%%%%%%%%%%
\begin{frame}[fragile]
\frametitle{Detalles importantes}

\begin{block}{Promoción}
\begin{itemize}
\item Una expresión compuesta sólo por enteros (incluyendo cualquier resultado intermedio) da como resultado un entero. 
\item La presencia de un número de punto flotante convierte a toda la expresión a punto flotante.
\item La presencia de un número complejo convierte a toda la expresión en compleja.
\end{itemize}
\end{block}
\begin{block}{Consejo}
Nunca está de más agregar paréntesis para hacer que las cosas sean más legibles aunque no sean estrictamente necesarios.
\end{block}
\end{frame}

%%%%%%%%%%%%%%%%%%%%%%%%%%%%%%%%%%%%%%%%%%%%%%%%%%%%%%%%%%
\begin{frame}[fragile]
\frametitle{Detalles importantes}

\begin{block}{Precedencia}

\begin{itemize}
\item La precedencia determina el orden en el que se llevan a cabo las operaciones en una {\em expresión}.  
\item La precedencia de operadores aritméticos sigue el orden PEMuDReS.
\item Primero paréntesis, luego exponenciación, luego multiplicación y división, luego resta y suma.
\item La exponenciación es asociativa hacia la derecha.
\end{itemize}
\end{block}
\end{frame}

%%%%%%%%%%%%%%%%%%%%%%%%%%%%%%%%%%%%%%%%%%%%%%%%%%%%%%%%%%
\begin{frame}[fragile]
\frametitle{Detalles importantes}

\begin{block}{Precedencia}

\begin{itemize}
\item La precedencia determina el orden en el que se llevan a cabo las operaciones en una {\em expresión}.  
\item La precedencia de operadores aritméticos sigue el orden PEMuDReS.
\item Primero paréntesis, luego exponenciación, luego multiplicación y división, luego resta y suma.
\item La exponenciación es \alert{asociativa hacia la derecha}.
\end{itemize}
\end{block}
\end{frame}

%%%%%%%%%%%%%%%%%%%%%%%%%%%%%%%%%%%%%%%%%%%%%%%%%%%%%%%%%%
\begin{frame}[fragile]
\frametitle{Un adelanto: Algunas funciones}

\begin{block}{Algunas funciones útiles}

\begin{itemize}
\item \mintinline{python}{abs(x) }: Devuelve el valor absoluto.
\item \mintinline{python}{complex(x,y)}: Devuelve el complejo $x+iy$
\item \mintinline{python}{int(x)}: Devuelve la parte entera de $x$
\item \mintinline{python}{float(n)}: Convierte el argumento en punto flotante
\item \mintinline{python}{hex(n)}: Devuelve la representación hexadecimal del entero $n$
\item \mintinline{python}{oct(n)}: Devuelve la representación octal de entero $n$
\item \mintinline{python}{bin(n)}: Devuelve la representación binaria del entero $n$
\item \mintinline{python}{round(n,[m])}: Redondea al flotante más cercano con $m$ dígitos después de la coma o al entero más cercano si $m$ está ausente.
\end{itemize}
\end{block}
\end{frame}

%%%%%%%%%%%%%%%%%%%%%%%%%%%%%%%%%%%%%%%%%%%%%%%%%%%%%%%%%%%
\begin{frame}
\frametitle{Síntesis y recursos:}

\begin{itemize}
\item \href{https://docs.python.org/3/reference/index.html}{Manual de referencia de Python}
\item \href{https://docs.python.org/3/library/index.html}{Manual de la Librería estándar de Python}
\end{itemize}
\end{frame}

\end{document}

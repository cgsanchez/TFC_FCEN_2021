\documentclass{beamer}

\usepackage[utf8]{inputenc}
\usepackage[spanish]{babel}
\usepackage[outputdir=.build]{minted}
\usepackage{hyperref}
\usepackage{graphicx}

\hypersetup{
    colorlinks = true
}

\usetheme{Madrid}

%Information to be included in the title page:
\title{Taller de Física Computacional}
\subtitle{NumPy - (a vuelo rasante)}
\author{Cristián G. Sánchez y Carlos J. Ruestes}
\date{2021}

\begin{document}

\frame{\titlepage}

%%%%%%%%%%%%%%%%%%%%%%%%%%%%%%%%%%%%%%%%%%%%%%%%%%%%%%%%%%
\begin{frame}[fragile]
    \frametitle{Numpy}
 
    \begin{block}{Listas}
        \begin{itemize}
            \item Las listas son contenedores abstractos para secuencias de objetos de cualquier tipo.
            \item Las listas son versátiles.
            \item En computación científica, en general, se trabaja con secuencias ordenadas de números del mismo tipo.
            \item La versatilidad se transforma en un obstáculo, las operaciones sobre listas son {\em lentas} 
        \end{itemize}
    \end{block}

    \end{frame}

%%%%%%%%%%%%%%%%%%%%%%%%%%%%%%%%%%%%%%%%%%%%%%%%%%%%%%%%%%
\begin{frame}[fragile]
    \frametitle{Numpy}
 
    \begin{block}{{\tt ndarray}}
        Para facilitar las operaciones sobre secuencias ordenadas de números Numpy implementa una nueva clase
        (objeto) llamado {\tt ndarray} junto con una serie de métodos y funciones que aceleran estas operaciones.
        \begin{itemize}
            \item Un {\tt ndarray} representa un arreglo multidimensional (vector, matriz, tensor, etc.)
            {\bf homogéneo}
            \item Con homogéneo nos referimos a que un {\tt ndarray} contiene elementos que son {\bf todos del mismo tipo.}
            \item Los tipos que pueden formar parte de un {\tt ndarray} deben poder ser descriptos por un número fijo de bytes.
            \item Los elementos de un {\tt ndarray} se almacenan (físicamente) en la RAM en un único bloque de memoria.
        \end{itemize}
    \end{block}

    \end{frame}

%%%%%%%%%%%%%%%%%%%%%%%%%%%%%%%%%%%%%%%%%%%%%%%%%%%%%%%%%%
\begin{frame}[fragile]
    \frametitle{{\tt numpy.dtype}}
    Los elementos de un {\tt ndarray} son {\tt dtype}s. 
    NumPy implementa una serie de tipos numéricos mucho más amplia que Python genérico.
    Algunos ejemplos son:
    \vspace{0.3cm}
    \begin{minted}[autogobble=true]{python}
    np.bool         np.csingle      np.int_
    np.single       np.cdouble      np.long
    np.double       np.clogdouble   np.longlong
    np.longdouble   np.intc         np.byte
    \end{minted}
    \vspace{0.3cm}    
    Los usuarios pueden definir nuevos {\tt dtype}s.
    \end{frame}

%%%%%%%%%%%%%%%%%%%%%%%%%%%%%%%%%%%%%%%%%%%%%%%%%%%%%%%%%%
\begin{frame}[fragile]
    \frametitle{Creación de un {\tt ndarray}}
    Antes que nada importar Numpy. La forma usual es \mintinline{python}{import numpy as np}
    \begin{block}{Algunas formas de crear {\tt ndarray}s}
        \begin{itemize}
            \item \mintinline{python}{A = np.empty((5,5))} 
            \item \mintinline{python}{A = np.zeros((5,5))} 
            \item \mintinline{python}{A = np.ones((5,5))}  
            \item \mintinline{python}{A = np.zeros_like(B)}
            \item \mintinline{python}{A = np.array([[0,1],[2,3]])}
            \item \mintinline{python}{A = np.zeros((1000,100),dtype=np.cdouble)} 
        \end{itemize}
    \end{block}
    \end{frame}

%%%%%%%%%%%%%%%%%%%%%%%%%%%%%%%%%%%%%%%%%%%%%%%%%%%%%%%%%%
\begin{frame}[fragile]
    \frametitle{Útiles para crear arreglos con grillas}
    \begin{block}{grillas en $\mathbb{R}$}
        \begin{itemize}
            \item \mintinline{python}{A = np.arrange(0,10,0.1)} Vector de números equiespaciados en 0.1 entre 0 y 10
            \item \mintinline{python}{A = np.linspace(0,2*np.pi,1000)} Vector de 1000 números equiespaciados entre 0 y $2\pi$
        \end{itemize}
    \end{block}
    \begin{block}{grillas en $\mathbb{R}^n$}
        \begin{minted}[autogobble=true]{python}
            xs = np.linspace(0,2*np.pi,100)
            xy = np.linspace(0,2*np.pi,100)
            xm,ym = np.meshgrid(xs,xy)
        \end{minted}
    \end{block}
    \end{frame}


%%%%%%%%%%%%%%%%%%%%%%%%%%%%%%%%%%%%%%%%%%%%%%%%%%%%%%%%%%
\begin{frame}[fragile]
    \frametitle{Indexación}
    \begin{block}{Indexación}
        \begin{itemize}
            \item Los elementos de un {\tt ndarray} se indexan como una secuencia pero con una {\em tupla} de la forma \mintinline{python}{A[1,2,3]}.
            \item Para cada índice de la tupla se pueden utilizar todas las formas de indexación que vimos para las secuencias.
            \item Ejemplos:
            \begin{minted}[autogobble=true]{python}
                A[0,0]
                A[-1,-1] 
                A[0:3,4:10:1]
                A[:,2]
                A[2,:]
                C[0:10:2,2,:]
            \end{minted}
       \end{itemize}
    \end{block}
    \end{frame}

%%%%%%%%%%%%%%%%%%%%%%%%%%%%%%%%%%%%%%%%%%%%%%%%%%%%%%%%%%
\begin{frame}[fragile]
    \frametitle{Vectorización y {\em broadcasting}}
    \alert{Los loops sobre elementos de un {\tt ndarray} son lentos.} Esta es una limitación intrínseca de Python 
    (que ya hemos discutido). Numpy implementa para los objetos del tipo {\tt ndarray} métodos que permiten hacer
    operaciones que uno normalmente haría en un loop de otra forma:
    \begin{block}{{\em broadcasting}}
        Las operaciones entre arreglos y escalares (o arreglos compatibles) se hacen elemento a elemento
        para todos los elementos del/los arreglo/s.$\dagger$
    \end{block}
    \begin{block}{{\em vectorización}}
        Una función (vectorizada) aplicada a un arreglo opera sobre todos los elementos, elemento a elemento.
    \end{block}
    $\dagger$ Para comprender los detalles de la compatibilidad
    \href{https://numpy.org/doc/stable/user/basics.broadcasting.html}{leer el manual}.
    \end{frame}

%%%%%%%%%%%%%%%%%%%%%%%%%%%%%%%%%%%%%%%%%%%%%%%%%%%%%%%%%%
\begin{frame}[fragile]
    \frametitle{{\em broadcasting}}
    Sean $A$, $B$ y $C$ arreglos de tres índices con dimensiones $L$, $M$ y $N$. Queremos hacer:
    \vspace{0.3cm}
    \begin{minted}[autogobble=true]{python}
    for i in range(L):
        for j in range(M):
            for k in range(N):
                # OJO! mismos índices
                C[i,j,k] = (3*(A[i,j,k]**2) + 1.0) * B[i,j,k] 
    \end{minted}
    \vspace{0.3cm}
    En NumPy esto se hace de la forma:
    \vspace{0.3cm}
    \begin{minted}[autogobble=true]{python}
    C = (3*(A**2) + 1.0) * B # 300 veces más rápido con L=M=N=100
    \end{minted}
    \end{frame}

%%%%%%%%%%%%%%%%%%%%%%%%%%%%%%%%%%%%%%%%%%%%%%%%%%%%%%%%%%
\begin{frame}[fragile]
    \frametitle{vectorización}
    Sean $A$, $C$ arreglos de tres índices con dimensiones $L$, $M$ y $N$. Queremos hacer:
    \vspace{0.3cm}
    \begin{minted}[autogobble=true]{python}
    for i in range(L):
        for j in range(M):
            for k in range(N):
                # OJO! mismos índices
                C[i,j,k] = m.exp(m.sin(m.cos(A[i,j,k]))) 
    \end{minted}
    \vspace{0.3cm}
    En NumPy esto se hace de la forma:
    \vspace{0.3cm}
    \begin{minted}[autogobble=true]{python}
    C = np.exp(np.sin(np.cos(A))) # 26 veces más rápido 
    \end{minted}
    \end{frame}

%%%%%%%%%%%%%%%%%%%%%%%%%%%%%%%%%%%%%%%%%%%%%%%%%%%%%%%%%%
\begin{frame}[fragile]
    \frametitle{vectorización}
    Para una función definida por el usuario:
    \vspace{0.3cm}
    \begin{minted}[autogobble=true]{python}
    def f(x):
        return x**2 + 1
    
    for i in range(L):
        for j in range(M):
            for k in range(N):
            # OJO! mismos índices
                C[i,j,k] = f(A[i,j,k]) 
    \end{minted}
    \vspace{0.3cm}
    En NumPy esto se hace de la forma:
    \vspace{0.3cm}
    \begin{minted}[autogobble=true]{python}
    f_vec = np.vectorize(f)
    C = f_vec(A) # 4 veces más rápido
    \end{minted}
    \end{frame}

%%%%%%%%%%%%%%%%%%%%%%%%%%%%%%%%%%%%%%%%%%%%%%%%%%%%%%%%%%%
\begin{frame}
\frametitle{Síntesis y recursos:}

\begin{itemize}
\item \href{https://numpy.org/doc/stable/}{Documentación de NumPy}
\item \href{https://numpy.org/doc/stable/reference/arrays.html}{Objetos {\tt ndarray}}
\item \href{https://numpy.org/doc/stable/user/basics.broadcasting.html}{Bradcasting basics}
\item \href{https://numpy.org/doc/stable/user/theory.broadcasting.html}{Broadcasting theory}
\item Machete de NumPy que viene en el material de esta clase.

\end{itemize}
\end{frame}

\end{document}

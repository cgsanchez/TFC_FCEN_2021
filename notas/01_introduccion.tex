\documentclass[aspectratio=169]{beamer}

\usepackage[utf8]{inputenc}
\usepackage[spanish]{babel}
\usepackage[outputdir=.build]{minted}
\usepackage{hyperref}
\usepackage{graphicx}

\usetheme{Madrid}

%Information to be included in the title page:
\title{Taller de Física Computacional}
\subtitle{Generales e Introducción}
\author{Cristián G. Sánchez y Carlos J. Ruestes}
\date{2021}

\begin{document}

\frame{\titlepage}

%%%%%%%%%%%%%%%%%%%%%%%%%%%%%%%%%%%%%%%%%%%%%%%%%%%%%%%%%%

\begin{frame}
\frametitle{Condiciones de contorno}
\begin{block}{Recursos}
\begin{itemize}
\item Grupo de Telegram: \url{https://t.me/tfcfcen}
\item Google Classroom: (código a enviar)
\item Clases de \alert{consulta}: miércoles y viernes de 10 a 12.
\end{itemize}
\end{block}
\end{frame}

%%%%%%%%%%%%%%%%%%%%%%%%%%%%%%%%%%%%%%%%%%%%%%%%%%%%%%%%%%

\begin{frame}
\frametitle{Expectativas de logro}
\begin{itemize}
\item Dominar los conceptos básicos para el manejo del lenguaje Python. 
\item Dominar los siguientes elementos de programación: Algoritmos y estructuras de datos simples, así como modelos de control de flujo de ejecución. 
\item Aplicar programación a la resolución de problemas de distintas ramas de la física mediante el uso de computadoras utilizando lenguaje Python como herramienta. 
\item Diseñar y desarrollar visualizaciones efectivas de resultados en gráficos bi y tridimensionales.
\end{itemize}
\end{frame}

%%%%%%%%%%%%%%%%%%%%%%%%%%%%%%%%%%%%%%%%%%%%%%%%%%%%%%%%%%

\begin{frame}
\frametitle{¿Cómo desarrollaremos el curso? ({\em \href{https://dle.rae.es/desiderata}{desiderata}})}
Con los siguientes elementos
\begin{itemize}
\item ``Pedacitos'' de ``teoría''.
\item Ejemplos resueltos en forma demostrativa.
\item Ejercitación.
\item Clases de consulta.
\end{itemize}
\end{frame}

%%%%%%%%%%%%%%%%%%%%%%%%%%%%%%%%%%%%%%%%%%%%%%%%%%%%%%%%%%

\begin{frame}
\frametitle{¿Porqué Python?}

A favor:
\begin{itemize}
\item Es un lenguaje \alert{fácil de entender} (ponele).
\item Es un lenguaje \alert{flexible}.
\item Es un lenguaje \alert{multipropósito}.
\item Es un lenguaje \alert{de alto nivel}.
\item Es un lenguaje \alert{interpretado}.
\end{itemize}

\end{frame}

%%%%%%%%%%%%%%%%%%%%%%%%%%%%%%%%%%%%%%%%%%%%%%%%%%%%%%%%%%%

\begin{frame}
\frametitle{¿Porqué Python?}

En contra:
\begin{itemize}
\item Es \alert{lento}.
\item No es amigable con \alert{multiprocesamiento}.
\item Al usar usar {\em tipos dinámicos} para las variables mucho depende del contexto.
\item Puede ser contraintuitivo para los que vienen de lenguajes más ``duros''.
\item Es \alert{enorme}, la misma cosa se puede hacer de \alert{muchas, muchas} formas diferentes.
\end{itemize}
\end{frame}

%%%%%%%%%%%%%%%%%%%%%%%%%%%%%%%%%%%%%%%%%%%%%%%%%%%%%%%%%%%

\begin{frame}
\frametitle{Lenguajes compilados e interpretados}
\begin{block}{Un lenguaje compilado}
Se utiliza un {\em compilador} para pasar del lenguaje (C por ejemplo) a un programa en {\em lenguaje de máquina} que luego es ejecutado.
\end{block}
\vspace{0.5cm}
\begin{block}{Un lenguaje interpretado}
Un programa {\em intérprete} ejecuta porciones de {\em lenguaje de máquina} \alert{preestablecidos} a medida que interpreta.
\end{block}
\end{frame}

%%%%%%%%%%%%%%%%%%%%%%%%%%%%%%%%%%%%%%%%%%%%%%%%%%%%%%%%%%%

\begin{frame}
\frametitle{El tipo de las variables es dinámico}
\begin{block}{Tipeado dinámico}
En Python el {\bf tipo} de las {\em variables} es \alert{dinámico}. 
\end{block}
A esta oración todavía le falta mucho contexto que le iremos agregando, pero es algo importante a tener en cuenta. A diferencia de otros lenguajes (paradigmáticamente {\tt C}) el tipo de una variable (entero, real, caracteres, etc.) puede ir cambiando a lo largo del programa y se determina de forma dinámica, siempre en base al programa claro. Esto le da al lenguaje mucha ``potencia'' ya que es posible escribir código compacto que hace muchas cosas ``detrás de la escena''. Como toda fortaleza puede convertirse en una debilidad si uno no tiene claro exactamente qué es lo que está haciendo.
\end{frame}

%%%%%%%%%%%%%%%%%%%%%%%%%%%%%%%%%%%%%%%%%%%%%%%%%%%%%%%%%%%

\begin{frame}
\frametitle{Síntesis ($\pm$)}
\begin{block}{Taller de Física Computacional}
Vamos a aprender (un subconjunto pequeño) del lenguaje Python \alert{haciendo y errando}, es decir programando (ese tema de la bicicleta).
\end{block}

\end{frame}

\end{document}

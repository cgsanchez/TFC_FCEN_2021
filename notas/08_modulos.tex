\documentclass{beamer}

\usepackage[utf8]{inputenc}
\usepackage[spanish]{babel}
\usepackage[outputdir=.build]{minted}
\usepackage{hyperref}
\usepackage{graphicx}

\hypersetup{
    colorlinks = true
}

\usetheme{Madrid}

%Information to be included in the title page:
\title{Taller de Física Computacional}
\subtitle{Nociones de Módulos}
\author{Cristián G. Sánchez y Carlos J. Ruestes}
\date{2021}

\begin{document}

\frame{\titlepage}

%%%%%%%%%%%%%%%%%%%%%%%%%%%%%%%%%%%%%%%%%%%%%%%%%%%%%%%%%%
\begin{frame}[fragile]
\frametitle{Módulos}
\begin{block}{{\em Módulos}}
De acuerdo al \href{https://docs.python.org/3/glossary.html}{Glosario de Python}:
\begin{quote}
Un {\em módulo} es un objeto que sirve como una unidad organizacional de código Python. Los módulos tienen un {\em espacio de nombres} que contienen objetos de Python arbitrarios. Los módulos se cargan en Python a través del proceso de {\em importación}.
\end{quote}
\end{block}
\begin{itemize}
\item En la práctica los módulos permiten construir bibliotecas de funciones, clases, variables con valores, etc..
\item Los módulos se escriben en un archivo {\tt nombre.py} que debe residir en una carpeta accesible a Python.
\item Los módulos pueden organizarse en paquetes que pueden a su vez contener otros sub-módulos y/o sub-paquetes.
\end{itemize}

\end{frame}

%%%%%%%%%%%%%%%%%%%%%%%%%%%%%%%%%%%%%%%%%%%%%%%%%%%%%%%%%%
\begin{frame}[fragile]
\frametitle{Módulos}
\begin{block}{{\em Módulos}}
La \href{https://docs.python.org/3/library/index.html}{Biblioteca estándar} de python contiene decenas de módulos entre los que podemos nombrar por ejemplo:
\begin{itemize}
\item {\tt math} Implementa funciones trascendentes y otras funciones matemáticas.
\item {\tt cmath} Implementa funciones matemáticas definidas para argumentos complejos.
\item {\tt os} Implementa interfases con el sistema operativo.
\item Hay otros para el manejo de fechas, cadenas de caracteres, almacenamiento de objetos, compresión, criptografía, etc..
\end{itemize}
\end{block}
\end{frame}

%%%%%%%%%%%%%%%%%%%%%%%%%%%%%%%%%%%%%%%%%%%%%%%%%%%%%%%%%%
\begin{frame}[fragile]
\frametitle{Módulos para programación científica}
Los paquetes que más utilizaremos en el curso son:
\begin{block}{{\em NumPy}}
\href{https://numpy.org/}{NumPy}:
Biblioteca de funciones y clases para cálculo numérico, en particular para arreglos multidimensionales. Muy recientemente, luego de 15 años de existencia, el equipo de desarrollo NumPy tiene un \href{https://www.nature.com/articles/s41586-020-2649-2}{paper en nature}.
\end{block}
\begin{block}{{\em matplotlib}}
\href{https://matplotlib.org}{matplotlib}:
Biblioteca de funciones y clases para visualización.
\end{block}
\begin{block}{{\em SciPy}}
\href{https://scipy.org}{SciPy}:
Una lista demasiado larga de funciones, clases, constantes y otros muy útiles en programación científica en general.
\end{block}
\end{frame}

%%%%%%%%%%%%%%%%%%%%%%%%%%%%%%%%%%%%%%%%%%%%%%%%%%%%%%%%%%
\begin{frame}[fragile]
\frametitle{Módulos para programación científica}
Otros proyectos importantes dentro del ecosistema de programación científica en Python incluyen \href{https://pandas.pydata.org/}{pandas} y \href{https://www.sympy.org/en/index.html}{SymPy}. Estos paquetes implementan funcionalidades de estadística y análisis de datos y matemática simbólica respectivamente. Estos proyectos, entre otros, son financiados por la ONG \href{https://numfocus.org/sponsored-projects}{NumFOCUS}.

\end{frame}

%%%%%%%%%%%%%%%%%%%%%%%%%%%%%%%%%%%%%%%%%%%%%%%%%%%%%%%%%%
\begin{frame}[fragile]
    \frametitle{Sintaxis}
    \begin{block}{Sintaxis}
    Para importar un módulo o un módulo parte de un paquete con todos sus métodos pueden usarse las sintaxis:
    \begin{minted}{python}
    import modulo 
    from paquete import módulo
    \end{minted}
    o
    \begin{minted}{python}
    import modulo as alias
    from paquete import módulo as alias
    \end{minted}
    En el caso de usar un alias este reemplaza el nombre del módulo para llamar a sus métodos.
    \end{block}
    \end{frame}

%%%%%%%%%%%%%%%%%%%%%%%%%%%%%%%%%%%%%%%%%%%%%%%%%%%%%%%%%%
\begin{frame}[fragile]
    \frametitle{Sintaxis}
    \begin{block}{Sintaxis}
    Para importar un método o conjunto de métodos de un módulo:
    \begin{minted}{python}
    from modulo import método1, método2, método3
    \end{minted}
    Lo mismo vale para importar módulos enteros de un paquete, etc..

    Para importar todos los métodos de un módulo
    \begin{minted}{python}
    from modulo import *
    \end{minted}
    Si bien la primera forma se usa, la segunda no se recomienda porque ``pisa'' todo el espacio
    de nombres en el cual se hace la importación, es decir, se sobreescriben todos los nombres que haya ya
    definidos que sean iguales a los del módulo.
    \end{block}
    \end{frame}

%%%%%%%%%%%%%%%%%%%%%%%%%%%%%%%%%%%%%%%%%%%%%%%%%%%%%%%%%%
\begin{frame}[fragile]
\frametitle{Sintaxis}
\begin{block}{Sintaxis}
Los métodos, clases, etc. de un módulo se acceden usando la notación que vimos para las propiedades y métodos de un objeto:
\begin{minted}{python}
modulo.funcion(parametro)
\end{minted}
o
\begin{minted}{python}
alias.función(parámetro)
\end{minted}
\end{block}
\end{frame}

%%%%%%%%%%%%%%%%%%%%%%%%%%%%%%%%%%%%%%%%%%%%%%%%%%%%%%%%%%%
\begin{frame}
\frametitle{Síntesis y recursos:}

\begin{itemize}
\item \href{https://numpy.org/doc/stable/}{Documentación de NumPy}
\item \href{https://numpy.org/doc/stable/reference/routines.math.html}{Funciones matemáticas en NumPy}

\end{itemize}
\end{frame}

\end{document}
